\subsection{Траектория и динамика движения}
\label{sub:domain:subsec_trajectory}
%cite https://ru.wikipedia.org/wiki/Траектория

Траектория материальной точки — линия в пространстве, по которой движется тело, представляющая собой множество точек, в которых находилась, находится или будет находиться материальная точка при своём перемещении в пространстве относительно выбранной системы отсчёта. 

Возможно наблюдение траектории при неподвижности объекта, но при движении системы отсчета. Так, звёздное небо может послужить хорошей моделью инерциальной и неподвижной системы отсчёта. Однако при длительной экспозиции эти звёзды представляются движущимися по круговым траекториям.

Возможен и случай, когда тело явно движется, но траектория в проекции на плоскость наблюдения является одной неподвижной точкой. Это, например, случай летящей прямо в глаз наблюдателя пули или уходящего от него поезда.

Принято описывать траекторию материальной точки в наперед заданной системе координат при помощи радиус-вектора, направление, длина и начальная точка которого зависят от времени. При этом кривая, описываемая концом радиус-вектора в пространстве может быть представлена в виде сопряженных дуг различной кривизны, находящихся в общем случае в пересекающихся плоскостях. При этом кривизна каждой дуги определяется её радиусом кривизны, направленном к дуге из мгновенного центра поворота, находящегося в той же плоскости, что и сама дуга. При том прямая линия рассматривается как предельный случай кривой, радиус кривизны которой может считаться равным бесконечности. И потому траектория в общем случае может быть представлена как совокупность сопряженных дуг.

Существенно, что форма траектории зависит от системы отсчета, избранной для описания движения материальной точки. Так, прямолинейное равномерно ускоряющееся движение в одной инерциальной системе в общем случае будет параболическим в другой равномерно движущейся инерциальной системе отсчёта.

%вставить сюда раздел Движение под действием внешних сил в инерциальной системе отсчёта с формулами

Действующие на материальную точку силы в этом понимании однозначно определяют форму траектории её движения (при известных начальных условиях). Обратное утверждение в общем случае не справедливо, поскольку одна и та же траектория может иметь место при различных комбинациях активных сил и реакций связи.

%cite https://ru.wikipedia.org/wiki/Динамика_(физика)
Динамика — раздел механики, в котором изучаются причины возникновения механического движения. Динамика оперирует такими понятиями, как масса, сила, импульс, момент импульса, энергия.

Также динамикой нередко называют, применительно к другим областям физики (например, к теории поля), ту часть рассматриваемой теории, которая более или менее прямо аналогична динамике в механике, противопоставляя обычно кинематике (к кинематике в таких теориях обычно относят, например, соотношения, получающиеся из преобразований величин при смене системы отсчёта).

Иногда слово динамика применяется в физике и не в описанном смысле, а в более общелитературном: для обозначения просто процессов, развивающихся во времени, зависимости от времени каких-то величин, не обязательно имея в виду конкретный механизм или причину этой зависимости.

Динамика, базирующаяся на законах Ньютона, называется классической динамикой. Классическая динамика описывает движения объектов со скоростями от долей миллиметров в секунду до километров в секунду.

Однако эти методы перестают быть справедливыми для движения объектов очень малых размеров (элементарные частицы) и при движениях со скоростями, близкими к скорости света. Такие движения подчиняются другим законам.

С помощью законов динамики изучается также движение сплошной среды, т. е. упруго и пластически деформируемых тел, жидкостей и газов.

В результате применения методов динамики к изучению движения конкретных объектов возник ряд специальных дисциплин: небесная механика, баллистика, динамика корабля, самолёта и т. п.

Эрнст Мах считал, что основы динамики были заложены Галилеем.

Исторически деление на прямую и обратную задачу динамики сложилось следующим образом.

Прямая задача динамики: по заданному характеру движения определить равнодействующую сил, действующих на тело.
Обратная задача динамики: по заданным силам определить характер движения тела.

%cite https://ru.wikipedia.org/wiki/Прямая_задача_динамики
Обратная задача динамики — определение координат тела и его скорости в любой момент времени по известным начальным условиям и силам, действующим на тело. Для ее решения необходимо знать координаты и скорость тела в некоторый начальный момент времени и силу, действующую на тело в любой последующий момент времени.

Силы в механике зависят от координат и скоростей движения тела. Для нахождения координат тела в любой момент времени необходимо по известным значениям сил, действующих на тело, и известной массе тела, согласно второму закону Ньютона, определить его ускорение, а затем последовательным интегрированием ускорения аналитическими или численными методами найти новое значение скорости тела, его перемещение и координаты. Обратную задачу механики часто приходится решать инженерам при проектировании машин и механизмов.

Например, при расчете траектории космического корабля на основе знания начальных условий и гравитационных сил, действующих на него со стороны планет, необходимо решить прямую задачу механики. Зная силу взаимодействия гребного винта с водой и силу сопротивления воды движению корпуса судна, можно определить, как будет двигаться судно, какую скорость оно может развить.

Существуют и задачи динамики смешанного типа, например, вычисление движения тел с наложенными на них связями. В таких случаях задача сводится не только к определению движения каждой материальной точки системы, но и к нахождению сил реакций связей

