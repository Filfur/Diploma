\sectioncentered*{Введение}
\addcontentsline{toc}{section}{Введение}
\label{sec:intro}
Возможности человеческого мозга уникальны. Благодаря накоплению жизненного опыта, умению быстро анализировать поступающие потоки данных и наблюдению за окружающим миром, он может предсказывать многие ситуации и ожидать конкретного поведения объектов в реальном мире. Так, профессиональный баскетболист, совершая дальний бросок из трехочковой зоны, за доли секунды просчитывает какую силу надо приложить к мячу и в каком направлении его бросить, исходя из веса мяча, собственных физических возможностей и расстояния до баскетбольной корзины. Правильно просчитанный бросок позволяет ему получить заветные три очка для его команды. Но даже если человек не является профессиональный баскетболистом, а просто любит играть в мобильные игры на телефоне по дороге на работу или учебу, или же он увлекается спортивными мобильными симуляторами, где бросает виртуальный мяч в корзину, то ему очень важно, чтобы этот мяч летел по всем законам физики и игра максимально симулировала реальный бросок мяча. 

Для получения положительного игрового опыта важно, чтобы физический движок игры корректно просчитывал все физические взаимодействия игровых объектов. Если мяч полетит не в ожидаемую сторону и будет вести себя не так, как мяч в реальном мире, то игрок будет разочарован и может просто удалить игру, а создатель данной игры потеряет пользователя и возможную прибыль. Поэтому в играх очень важна правильная симуляция физики объектов.

Одним из важных элементов игрового процесса также является построение траектории движения объекта. Если пользователь, играя в симулятор бильярда, загнает шар в лузу с отскоком от борта стола, то игра построит ему траекторию отскока шара от борта, что позволит шару правильно выбрать точку удара и достичь своей цели. Симулятор баскетбола построит параболическую траекторию движения мяча. Симулятор снайпера построит баллистическую траекторию движения пули.  В некоторых играх пользователю надо показать как будет двигаться объект после большого числа столкновений с другими объектами. 

На данный момент многие физические движки дают реалистичную симуляцию поведения объектов в игровом мире. В основу большинства положен расчет изменения положений объектов каждый фрейм игры за счёт векторов сил действующих на объекты, какими являются линейная скорость объекта, гравитация и другие воздействующие силы. Так, шар, брошенный от уровня земли под определённым углом, с некоторой силой будет лететь по параболической траектории и в некоторый момент коснётся земли. 

Построение траектории объекта накладывает некоторые ограничения. Так, если каждый фрейм симулировать игру на несколько десятков секунд вперед и сохранять данные о перемещениях объектов для того, чтобы показать игроку траекторию движения объектов, исходя из изначальных сил, действующих на эти объекты, то это повлечет за собой уменьшение производительности игры, возможные притормаживания для выполнения всех расчетов и получение негативного опыта у пользователя. Поэтому разработчикам необходимо искать другие, более оптимальные пути решения этой проблемы.
 
В данном дипломном проекте ставятся следующие задачи:
\begin{itemize} 
	\item рассмотреть существующие игровые физические движки;
	\item ознакомится с особенностями и возможностями игровых движков Unity и libGDX;
	\item разработать программный модуль для построения траектории движения в двумерном пространстве, используя игровые движки Unity и libGDX, провести сравнения, выделить положительные и отрицательные стороны каждого из движков;
	\item разработать программный модуль для построения траектории движения в трехмерном пространстве.
\end{itemize}

В результате получился программный модуль, написанный на языке \csharp~с использованием среды разработки Unity, который можно использовать для построения траектории движения объектов в двумерном и трехмерном пространстве.
