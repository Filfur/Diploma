\subsection{История развития игровых движков}
\label{sub:domain:game_engint_hist}
%cite https://habr.com/ru/company/miip/blog/314502/
Вместе с созданием первых игр программисты пришли к тому, что каждая игра содержит общие компоненты, даже несмотря на различие аппаратных платформ. А первые игры имели место на игровых автоматах размером с холодильник.

Общая для игр функциональность — графические решения, игровые механики, расчет физики и другое — стала выделяться в отдельные библиотеки, но, для того чтобы быть «игровым движком» было еще далеко. Во многом это было связано с серьезным различием программно-аппаратных платформ и неопределенности в самих играх. Ведь жанры и типы игр еще предстояло изобрести, при том, что многие первые игры были текстовыми. Собственно, именно для ранних адвенчур и платформеров и стали возникать игровые движки, особенно с развитием графики — хорошим примером можно назвать Adventure Game Interpreter (AGI). При разработке King’s Quest в далеком 1984 году, программисты Sierra On-Line столкнулись с неудобством низкоуровневой разработки столь сложной и перспективной по графике в те времена игры — и разработали набор решений, которым и стал AGI. Всего на нем было выпущено 14 различных игр за 5 лет на 7 различных платформах, поэтому понятие “кроссплатформенность” было важным уже тогда.

Однако, движки того времени редко выходили за пределы изначальной компании-разработчика и, как правило, были достаточно узкоспециализированными под конкретный жанр игры.

Ситуация начала меняться в 1993-м году после выхода игры Doom от компании id Software. Хотя при ее разработке использовались наработки движка Wolfenstein 3D, с точки зрения возможностей и модульности в ней был совершен настоящий технологический прорыв. В то время видеопроцессоры были не способны эффективно работать с трехмерной графикой, поэтому Джон Кармак (ведущий программист движка) выполнял все необходимые математические вычисления, служащие для манипуляции с трехмерными объектами, светом, затенением, наложением текстур и прочего самостоятельно. В результате, изображение выглядело трехмерным, на самом деле таковым не являясь. Поэтому Doom engine (первая версия id Tech) был не истинно трехмерным, а псевдотрехмерным. Но важно то, что техническая составляющая этой игры задала стандарт для того, что могло называться игровым движком. А именно, движок Doom был модульным, представлял из себя набор подсистем, в нем каждый четко отделенный программный слой отвечал за обработку своей порции данных. В результате, использовать его для различных игр (Hexen, Heretic, Strife) и силами сторонних разработчиков (Raven Software и Rogue Entertainment) стало намного проще. Поэтому появление игровых движков относят к середине 90-х годов 20-го века, то есть тогда окончательно сформировалось определение игрового движка в современном смысле.

Игровой движок представляет своеобразную узкоспециализированную операционную систему, поскольку включает все модули последней. В него входят: система управления памятью, графическая подсистема, система ввода, аудио подсистема, искусственный интеллект, физическая подсистема, сетевая подсистема, редактор игровых уровней и другое. Кроме того ядро движка может предоставлять особый подход к работе с файлами – файловую (ресурсную) систему, а так же отличающиеся от основной операционной системы средства работы с многопоточностью. Современный игровые движки вдобавок включают интерпретатор скриптового языка, заточенного для описания игровой логики, а нередко и полностью визуальный ее редактор. Его использование позволяет абстрагироваться от описания низкоуровневых команд и инструкций, а сконцентрироваться на геймплее. На этом составляющие движок компоненты не ограничиваются, их может быть как больше, так и меньше.

Игровой движок в первую очередь создается в целях упрощения и ускорения разработки. Поэтому включает средства для создания игрового мира – level-моделинга, импорта объектов, текстурирования, загрузки и анимации персонажей, создания визуальных эффектов, настройки физики и прочего.

Второй значительной целью разработки движка является кроссплатформенность или платформонезависимость разрабатываемой игры. То есть возможность ее запуска с минимально возможными изменениями. Совсем без изменений на другой платформе осуществить запуск игры не удастся из-за аппаратных различий, в том числе: размеров экрана, средств и способов управления и др.

Развитие игровых движков происходит вместе или под влиянием развития аппаратных и программных платформ, вместе с появлением новых игровых жанров и изменениями вкусов пользователей. Коротко говоря, развитием игровой индустрии в целом.

В середине 90-х после появления видеопроцессоров, способных обрабатывать трехмерную графику стали появляться программные интерфейсы, упрощающие ее разработку. Вслед за кроссплатформенным OpenGL на сцену в составе DirectX вышел Direct3D для Windows. Эти 2 визуализатора на много лет вперед определили способы графического вывода в играх.

В 1996-м году вышла игра Quake на Quake Engine. Этот движок оказал колоссальное влияние на игровую индустрию. Дерево движков, поулчивших свою жизнь благодаря Quake Engine представлено на рисунке~\ref{fig:domain:game_engint_hist:quake_tree}.

\begin{figure}[p]
	\noindent\centering
	\includegraphics[width=140mm]{game_engine_tree.png}  
	\caption{ Дерево движков, основанных на Quake Engine }
	\label{fig:domain:game_engint_hist:quake_tree}
\end{figure}

Почти до конца десятилетия на рынке промежуточного программного обеспечения для игр (другими словами, игровых движков) практически единолично ритм задавала id Software. Однако в 1998-м году компания Epic Games выпустила успешную игру Unreal на одноименном движке — с настоящим технологическим прорывом по уровню графики. Ведущим программистом движка стал основатель Epic Тим Суини. Тим наравне с Кармаком является наиболее значимой фигурой в истории движков игровой индустрии — и Unreal Engine в его 3 и 4 версиях очень популярен и сейчас. Год спустя от Epic вышла ставшая еще более популярной игра Unreal Tournament.

В это же самое время конкурирующая компания-разработчик – id Software выпустила мультиплеерную игру Quake 3 Arena (на движке id Tech 3), ровно как Unreal Tournament включающую сетевые баталии.

Эти две игры стали флагманами индустрии, определив ее развитие на годы вперед.

На рынке было не так много игроков. Поэтому их продукция была очень дорога, и флагманские движки лицензировались только достаточно крупными разработчиками,

Ситуация начала коренным образом меняться примерно в середине первого десятилетия 21-го века. Тогда на рынке и в свободном доступе стало появляться большое количество средств для разработки игр. Бизнес промежуточного ПО (middleware) стал набирать обороты. Сначала рынок заполнился графическими фреймворками: Ogre, DarkGDK и др., предоставляющие программисту высокоуровневую прослойку над графическим API. В то же время отличающиеся от игровых движков полным отсутствием внутриигровых редакторов.

Затем на рынок пришли полноценные игровые движки по ценам, уместным для небольшой инди-команды разработчиков, среди них: Torque 3D, Unity 3D, и многие другие. Даже стартовавшие как флагманские движки — например, CryEngine от Crytek и ранее упомянутый Unreal Engine — стали использовать намного более доступную ценовую политику и стали доступны даже начинающим разработчикам.

Важным трендом игровой индустрии стали казуальные игры. Эти, по своей сути, незамысловатые, но красочные, не требующие бешеного взаимодействия с клавиатурой и мышкой головоломки с технической точки зрения были проще трехмерных хардкорных шутеров, поэтому для их разработки не понадобилось сильной модификации универсальных движков. Но, зато, в индустрии появились новые игроки, такие как: Torque Game Builder, HGE и другие.

В это же время, благодаря World of Warcraft, в игровой индустрии стали очень популярны MMORPG — а параллельно многие жанры делали все большую ставку на мультиплеер. Целый ряд движков не смог предоставить пользователям новую функциональность для клиент-серверных приложений, поэтому они ушли в небытие. Другие движки были адаптированы для мультиплеерного мира путем разработки для них серверных решений, так для Unity 3D были разработаны Photon и SmartFox. Третий тип универсальных движков, изначально являясь клиент-серверным, не почувствовал изменений. К нему относится Torque 3D. Также на рынке появились новые движки, предназначенные для глобальных многопользовательских игр, например HeroEngine, BigWorld, объединяющие масштабируемое под тысячи игроков серверное решение и доступный конкретному игроку клиент.

На рынке еще с 90х существовали браузерные игры, а затем второе рождение им дали социальные сети. необходимость эффективно создавать игры для браузера не осталась незамеченной. Разработчики универсальных движков, например Torque 2D/3D, Unity 3D отреагировали на это довольно оперативно, выпустив плагины для браузеров, которые позволили отображать графику прямо в окне последних. Сначала популярность завоевал визуализатор на основе технологии Flash, но по целому ряду причин эта технология все больше теряет свою долю на рынке. Поэтому сейчас для визуализации в вебе часто используется библиотека для языка JavaScript — WebGL, которая позволяет создавать интерактивную 3D-графику. Однако, из-за недостатков языка, таких как отсутствие многопоточности, библиотека не может полноценно удовлетворить потребности игроделов. Ей на смену консорциумом W3C (куда входят: Microsoft, Google, Mozilla и др.) разрабатывается новый низкоуровневый бинарный компилируемый формат WebAssembly.

Под конец первого десятилетия 21-го века очень быстро развивались мобильные технологии. Как гром среди ясного неба появились мобильные устройства по мощности сопоставимые с ПК средней ценовой категории и способные запускать мощные игровые приложения со всеми спецэффектами, которыми обладали низкоуровневые графические интерфейсы. На что разработчики игровых движков ответили в некоторых случаях созданием специализированных конверторов, создающих нативный для конкретного оборудования код (как, например, Unity 3D), а в других — модернизировали свои продукты для кроссплатформенности (к примеру, Torque 2D, Cocos 2DX). Также, на рынке появились новые игроки, предлагающие кроссплатформенные движки для всего парка мобильных устройств, выполняющиеся со скоростью нативного кода. Примеры подобных средств: Corona SDK, Marmalade SDK, AGK (App Game Kit).

Также, возник целый ряд кроссплатформенных движков, позволяющих разработать игру при минимальном знании программирования. Примерами можно назвать Construct 2 и GameMaker Pro. Используя готовые решения и визуальные редакторы, можно быстро — иногда в течение нескольких часов — создавать простые игры. Это оказалось особенно распространенным на мобильном рынке, где распространение free2play модели и короткая игровая сессия сделали “простые” игры вполне успешным жанром.
