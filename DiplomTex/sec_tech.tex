\section{Используемые технологии}
\label{sub:domain:sec_tech}

\subsection{Язык \csharp}~
%cite https://ru.wikipedia.org/wiki/C_Sharp
Объектно-ориентированный язык программирования. Разработан в 1998—2001 годах группой инженеров компании Microsoft под руководством Андерса Хейлсберга и Скотта Вильтаумота как язык разработки приложений для платформы Microsoft .NET Framework. Впоследствии был стандартизирован как ECMA-334 и ISO/IEC 23270.

\csharp относится к семье языков с C-подобным синтаксисом, из них его синтаксис наиболее близок к C++ и Java. Язык имеет статическую типизацию, поддерживает полиморфизм, перегрузку операторов (в том числе операторов явного и неявного приведения типа), делегаты, атрибуты, события, свойства, обобщённые типы и методы, итераторы, анонимные функции с поддержкой замыканий, LINQ, исключения, комментарии в формате XML.

Переняв многое от своих предшественников — языков C++, Delphi, Модула, Smalltalk и, в особенности, Java — \csharp, опираясь на практику их использования, исключает некоторые модели, зарекомендовавшие себя как проблематичные при разработке программных систем, например, \csharp в отличие от C++ не поддерживает множественное наследование классов (между тем допускается множественное наследование интерфейсов).

\subsection{Среда разработки и редактор Unity}~
%cite https://ru.wikipedia.org/wiki/Unity_(игровой_движок)
Редактор Unity имеет простой Drag\&Drop интерфейс, который легко настраивать, состоящий из различных окон, благодаря чему можно производить отладку игры прямо в редакторе. Движок использует для написания скриптов \csharp. Ранее поддерживались также Boo (диалект Python, поддержку убрали в 5-й версии) и модификация JavaScript, известная как UnityScript (поддержка прекращена в версии 2017.1). Расчёты физики производит физический движок PhysX от NVIDIA. Графический API - DirectX (на данный момент DX 11)

Проект в Unity делится на сцены (уровни) — отдельные файлы, содержащие свои игровые миры со своим набором объектов, сценариев, и настроек. Сцены могут содержать в себе как, собственно, объекты (модели), так и пустые игровые объекты — объекты, которые не имеют модели («пустышки»). Объекты, в свою очередь содержат наборы компонентов, с которыми и взаимодействуют скрипты. Также у объектов есть название (в Unity допускается наличие двух и более объектов с одинаковыми названиями), может быть тег (метка) и слой, на котором он должен отображаться. Так, у любого объекта на сцене обязательно присутствует компонент Transform — он хранит в себе координаты местоположения, поворота и размеров объекта по всем трём осям. У объектов с видимой геометрией также по умолчанию присутствует компонент Mesh Renderer, делающий модель объекта видимой.

К объектам можно применять коллизии (в Unity т. н. коллайдеры — collider), которых существует несколько типов.

Также Unity поддерживает физику твёрдых тел и ткани, а также физику типа Ragdoll (тряпичная кукла). В редакторе имеется система наследования объектов; дочерние объекты будут повторять все изменения позиции, поворота и масштаба родительского объекта. Скрипты в редакторе прикрепляются к объектам в виде отдельных компонентов.

При импорте текстуры в Unity можно сгенерировать alpha-канал, mip-уровни, normal-map, light-map, карту отражений, однако непосредственно на модель текстуру прикрепить нельзя — будет создан материал, которому будет назначен шейдер, и затем материал прикрепится к модели. Редактор Unity поддерживает написание и редактирование шейдеров. Редактор Unity имеет компонент для создания анимации, но также анимацию можно создать предварительно в 3D-редакторе и импортировать вместе с моделью, а затем разбить на файлы.

Unity 3D поддерживает систему Level Of Detail (сокр. LOD), суть которой заключается в том, что на дальнем расстоянии от игрока высокодетализированные модели заменяются на менее детализированные, и наоборот, а также систему Occlusion culling, суть которой в том, что у объектов, не попадающих в поле зрения камеры не визуализируется геометрия и коллизия, что снижает нагрузку на центральный процессор и позволяет оптимизировать проект. При компиляции проекта создается исполняемый (.exe) файл игры (для Windows), а в отдельной папке — данные игры (включая все игровые уровни и динамически подключаемые библиотеки).

Движок поддерживает множество популярных форматов. Модели, звуки, текстуры, материалы, скрипты можно запаковывать в формат .unityassets и передавать другим разработчикам, или выкладывать в свободный доступ. Этот же формат используется во внутреннем магазине Unity Asset Store, в котором разработчики могут бесплатно и за деньги выкладывать в общий доступ различные элементы, нужные при создании игр. Чтобы использовать Unity Asset Store, необходимо иметь аккаунт разработчика Unity. Unity имеет все нужные компоненты для создания мультиплеера. Также можно использовать подходящий пользователю способ контроля версий. К примеру, Tortoise SVN или Source Gear.

В Unity входит Unity Asset Server — инструментарий для совместной разработки на базе Unity, являющийся дополнением, добавляющим контроль версий и ряд других серверных решений.

Как правило, игровой движок предоставляет множество функциональных возможностей, позволяющих их задействовать в различных играх, в которые входят моделирование физических сред, карты нормалей, динамические тени и многое другое. В отличие от многих игровых движков, у Unity имеется два основных преимущества: наличие визуальной среды разработки и межплатформенная поддержка. Первый фактор включает не только инструментарий визуального моделирования, но и интегрированную среду, цепочку сборки, что направлено на повышение производительности разработчиков, в частности, этапов создания прототипов и тестирования. Под межплатформенной поддержкой предоставляется не только места развертывания (установка на персональном компьютере, на мобильном устройстве, консоли и т. д.), но и наличие инструментария разработки (интегрированная среда может использоваться под Windows и Mac OS).

Третьим преимуществом называется модульная система компонентов Unity, с помощью которой происходит конструирование игровых объектов, когда последние представляют собой комбинируемые пакеты функциональных элементов. В отличие от механизмов наследования, объекты в Unity создаются посредством объединения функциональных блоков, а не помещения в узлы дерева наследования. Такой подход облегчает создание прототипов, что актуально при разработке игр.

В качестве недостатков приводятся ограничение визуального редактора при работе с многокомпонентными схемами, когда в сложных сценах визуальная работа затрудняется. Вторым недостатком называется отсутствие поддержки Unity ссылок на внешние библиотеки, работу с которыми программистам приходится настраивать самостоятельно, и это также затрудняет командную работу. Еще один недостаток связан с использованием шаблонов экземпляров (англ. prefabs). С одной стороны, эта концепция Unity предлагает гибкий подход визуального редактирования объектов, но с другой стороны, редактирование таких шаблонов является сложным. Также, WebGL-версия движка, в силу специфики своей архитектуры (трансляция кода из \csharp в \cpp и далее в JavaScript), имеет ряд нерешенных проблем с производительностью, потреблением памяти и работоспособностью на мобильных устройствах.

\subsection{Microsoft Visual Studio}~
%cite https://ru.wikipedia.org/wiki/Microsoft_Visual_Studio
Линейка продуктов компании Microsoft, включающих интегрированную среду разработки программного обеспечения и ряд других инструментальных средств. Данные продукты позволяют разрабатывать как консольные приложения, так и приложения с графическим интерфейсом, в том числе с поддержкой технологии Windows Forms, а также веб-сайты, веб-приложения, веб-службы как в родном, так и в управляемом кодах для всех платформ, поддерживаемых Windows, Windows Mobile, Windows CE, .NET Framework, Xbox, Windows Phone .NET Compact Framework и Silverlight.

Visual Studio включает в себя редактор исходного кода с поддержкой технологии IntelliSense и возможностью простейшего рефакторинга кода. Встроенный отладчик может работать как отладчик уровня исходного кода, так и отладчик машинного уровня. Остальные встраиваемые инструменты включают в себя редактор форм для упрощения создания графического интерфейса приложения, веб-редактор, дизайнер классов и дизайнер схемы базы данных. Visual Studio позволяет создавать и подключать сторонние дополнения (плагины) для расширения функциональности практически на каждом уровне, включая добавление поддержки систем контроля версий исходного кода (как, например, Subversion и Visual SourceSafe), добавление новых наборов инструментов (например, для редактирования и визуального проектирования кода на предметно-ориентированных языках программирования) или инструментов для прочих аспектов процесса разработки программного обеспечения (например, клиент Team Explorer для работы с Team Foundation Server).