\section{ТЕХНИКО-ЭКОНОМИЧЕСКОЕ ОБОСНОВАНИЕ ПРОГРАММНОГО МОДУЛЯ ДЛЯ РАСЧЕТА ТРАЕКТОРИИ ДВИЖЕНИЯ ОБЪЕКТОВ В ИГРОВЫХ ПРИЛОЖЕНИЯХ}

\subsection{Характеристика проекта}
Разработанный программный модуль представляет собой приложение предназначенное для расчета и построения траектории движения физических тел с использованием игрового движка Unity. 
 
Модуль, написанный с использованием движка Unity позволяет:
\begin{itemize}
	\item Рассчитывать траекторию и имитировать полет объекта из одной точки пространства в другую.
	\item Рассчитывать траекторию и имитировать полёт объекта с учетом большого количества упругих столкновений с другими объектами.
	\item Манипулировать физическими параметрами летящих объектов и полета в общем (скоростью полёта, количеством летящих объектов, размерами объектов, силой трения с другими объектами, гравитацией).
	%\item Хранить данные траектории полёта в локальной и удаленной базе данных.
	%\item Синхронизировать данные между локальной и удаленной базами данных при необходимости.
	%\item Настраивать физические параметры летящих объектов в удаленной базе данных при помощи программного интерфейса.
\end{itemize}

Разработка программного модуля осуществлялась в среде Unity c использованием Unity Editor и Visual Studio.

Полученный модуль может использоваться для построения траектории и имитации движения различных объектов в кроссплатформенных игровых приложениях, построенных в среде Unity. Модуль может быть загружен в Unity Asset Store для продажи другим разработчикам. Модуль не имеет ограничений на используемую платформу и может быть использован на любых платформах поддерживаемых движком Unity. 

Пользователями полученного программного модуля могут являться как отдельные разработчики, так и компании, нуждающиеся в построении траектории движения объектов в пространстве.

Также модуль может быть использован в научных целях для исследования движений различных тел при некоторых заданных параметрах движения и окружающей среды.

Разрабатываемый модуль позволяет уменьшить время, затрачиваемое на разработку приложений, нуждающихся в построении траектории объектов на плоскости и в пространстве.

Модуль имеет самостоятельное значение. Разработка выполняется по договору с компанией BYBN. Личные неимущественные права принадлежат разработчику, а имущественные (исключительные) права переходят компании BYBN (собственнику разработки).

Расчеты ТЭО выполнены на май 2020 года.

\subsection{Расчет прогнозного экономического эффекта от реализации программного средства вычислительной техники}
