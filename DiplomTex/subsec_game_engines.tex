\subsection{Популярные кроссплатформенные игровые движки}
\label{sub:domain:subsec_game_engines}
\subsubsection{Unity Engine}~
\label{sub:domain:subsec_game_engines:unity}
%cite https://ru.wikipedia.org/wiki/Unity_(игровой_движок)

Межплатформенная среда разработки компьютерных игр, разработанная американской компанией Unity Technologies. Unity позволяет создавать приложения, работающие на более чем 25 различных платформах, включающих персональные компьютеры, игровые консоли, мобильные устройства, интернет-приложения и другие. Выпуск Unity состоялся в 2005 году и с того времени идет постоянное развитие.

Основными преимуществами Unity являются наличие визуальной среды разработки, межплатформенной поддержки и модульной системы компонентов. К недостаткам относят появление сложностей при работе с многокомпонентными схемами и затруднения при подключении внешних библиотек.

На Unity написаны тысячи игр, приложений, визуализации математических моделей, которые охватывают множество платформ и жанров. При этом Unity используется как крупными разработчиками, так и независимыми студиями.


\subsubsection{Unreal Engine}~
\label{sub:domain:subsec_game_engines:unreal}
%cite https://ru.wikipedia.org/wiki/Unreal_Engine

Игровой движок, разрабатываемый и поддерживаемый компанией Epic Games. Первой игрой на этом движке был шутер от первого лица Unreal, выпущенный в 1998 году. Хотя движок первоначально был предназначен для разработки шутеров от первого лица, его последующие версии успешно применялись в играх самых различных жанров, в том числе стелс-играх, файтингах и массовых многопользовательских ролевых онлайн-играх. В прошлом движок распространялся на условиях оплаты ежемесячной подписки; с 2015 года Unreal Engine бесплатен, но разработчики использующих его приложений обязаны перечислять 5\% роялти от общемирового дохода с некоторыми условиями.

Написанный на языке C++, движок позволяет создавать игры для большинства операционных систем и платформ: Microsoft Windows, Linux, Mac OS и Mac OS X; консолей Xbox, Xbox 360, Xbox One, PlayStation 2, PlayStation 3, PlayStation 4, PSP, PS Vita, Wii, Dreamcast, GameCube и др., а также на различных портативных устройствах, например, устройствах Apple (iPad, iPhone), управляемых системой iOS и прочих. (Впервые работа с iOS была представлена в 2009 году, в 2010 году продемонстрирована работа движка на устройстве с системой webOS).

Для упрощения портирования движок использует модульную систему зависимых компонентов; поддерживает различные системы рендеринга (Direct3D, OpenGL, Pixomatic; в ранних версиях: Glide, S3, PowerVR), воспроизведения звука (EAX, OpenAL, DirectSound3D; ранее: A3D), средства голосового воспроизведения текста, распознавание речи[8][9][10], модули для работы с сетью и поддержки различных устройств ввода.

Для игры по сети поддерживаются технологии Windows Live, Xbox Live, GameSpy и прочие, включая до 64 игроков (клиентов) одновременно. Таким образом, движок адаптировали и для применения в играх жанра MMORPG (один из примеров: Lineage II).

\subsubsection{Cocos2D}~
\label{sub:domain:subsec_game_engines:cocos}
%cite https://ru.wikipedia.org/wiki/Cocos2d

Кросс-платформенный фреймворк, используемый для разработки интерактивных приложений и игр (преимущественно для мобильных устройств). Является открытым программным обеспечением. Cocos2d содержит множество ответвлений, таких как Cocos2d-ObjC, Cocos2d-x, Cocos2d-html5 и Cocos2d-XNA. Также в сообществе Cocos2d имеется несколько независимых редакторов, предназначенных для редактирования спрайтов, частиц, шрифтов и тайловых карт. Можно также упомянуть редакторы мира: CocosBuilder и CocoStudio.

Работа всех версий Cocos2D основана на использовании спрайтов. Спрайты можно рассматривать как простые 2D изображения, но также может быть контейнером для других спрайтов. В Cocos2D, расположенные вместе спрайты создают сцену, к примеру, уровень игры или главное меню. Спрайтами можно управлять на основе событий в исходном коде или как часть анимации. Над спрайтами можно проводить всевозможные действия: перемещать, поворачивать, масштабировать, изменять изображение и так далее.

Cocos2D обеспечивает базовые примитивы анимации, которые используют спрайты. Некоторые версии Cocos2D позволяют эффекты частиц и применение шейдерных фильтров (warp, ripple и тд.).

Cocos2D предоставляет примитивы для создания простых элементов графического интерфейса. Они включают в себя текстовые поля, надписи, меню, кнопки и другие распространённые элементы.

\subsubsection{CryEngine}~
\label{sub:domain:subsec_game_engines:cry}
%cite https://ru.wikipedia.org/wiki/CryEngine

Игровой движок, созданный немецкой частной компанией Crytek в 2002 году и первоначально используемый в шутере от первого лица Far Cry. «CryEngine» — коммерческий движок, который предлагается для лицензирования другим компаниям. С 30 марта 2006 года все права на движок принадлежат компании Ubisoft.

Движок был лицензирован компанией NCSoft для разрабатываемой MMORPG Aion: Tower of Eternity.

В конце сентября 2009 года братья Ерли, основатели Crytek, дали интервью великобританскому журналу Develop, в котором заявили, что изначально CryEngine не планировался для лицензирования сторонними компаниями. CryEngine планировался стать закрытым движком для сугубо внутреннего использования.

Игровой движок CryEngine — первый коммерческий движок Crytek. Его разработка была начата сразу же после основания компании. Движок первоначально разрабатывался как технологическая демонстрация для американской компании nVidia. Однако на выставке ECTS 2000 (англ. European Computer Trade Show — Европейская Компьютерная Выставка) Crytek произвела большое впечатление на всех больших издателей, посетителей и журналистов их технической демонстрацией, которая была показана в отделе nVidia. После этого на основе движка было решено создать 2 игры — «X-Isle» и «Engalus». Ни одна из этих игр так и не была выпущена.

2 мая 2002 года Crytek официально объявляет о том, что их игровой движок CryEngine полностью закончен и готов для лицензирования сторонними компаниями. Crytek также предлагает для лицензирования свою новую разработку — программу PolyBump.

26 марта 2004 года первая коммерческая компьютерная игра от Crytek и первая игра, использующая CryEngine — «Far Cry» -- отправилась к розничным продавцам.

Особенности:
\begin{itemize}
	\item CryEngine Sandbox: редактор игры в реальном времени, предлагающий обратную связь «Что Вы видите, то Вы и ИГРАЕТЕ».
	\item Рендерер: интегрированные открытые (англ. outdoor) и закрытые (англ. indoor) локации без швов. Также рендерер поддерживает OpenGL и DirectX 8/9, Xbox с использованием последних аппаратных особенностей, PS2 и GameCube, а также Xbox 360.
	\item Физическая система: поддерживает инверсную кинематику персонажей, транспортные средства, твёрдые тела, жидкость, тряпичные куклы (англ. rag doll), имитацию ткани и эффекты мягкого тела. Система объединена с игрой и инструментами.
	\item Инверсная кинематика персонажей и смешанная анимация: позволяет модели иметь множественные анимации для лучшей реалистичности.
	\item Система игрового искусственного интеллекта: включает командный интеллект и интеллект, определяемый скриптами. Возможность создания особенных врагов и их поведения, не касаясь кода C++.
	\item Интерактивная динамическая система музыки: музыкальные дорожки отвечают действиям игрока и ситуации и предлагают качество CD-диска с полным 5.1 звуковым окружением.
	\item Звуковое окружение и механизм SFS: способность точно воспроизвести звуки от природы с плавным сопряжением без шва между средами и внутренними/внешними местоположениями в системе Dolby Digital 5.1. аудио. Включает аудио поддержку EAX 2.0.
	\item Сетевая система «клиент-сервер»: Управляет всеми сетевыми подключениями для режима с несколькими игроками. Это — система сети с низким временем отклика, основанная на архитектуре клиент-сервер.
	\item Шейдеры: скриптовая система используется для комбинирования текстур по-разному для увеличения визуальных эффектов. Поддерживается реальное попиксельное освещение, ухабистые отражения, преломления, объёмные эффекты жара, анимированные текстуры, прозрачные компьютерные дисплеи, окна, пулевые отверстия, и некоторые другие эффекты.
	\item Ландшафт: Используется расширенная карта высот и сокращение полигонов для создания массивной, реалистической среды. Видимое расстояние может составить до 2 км, когда преобразовано из игровых модулей.
	\item Освещение и тени: комбинация предрасчётных теней и теней реального времени, стенсильные тени и lightmaps (карты теней) для улучшения динамического окружения. Включает правильную перспективу с высокой разрешающей способностью и объёмные гладко-теневые реализации для драматического и реалистического внутреннего затенения. Поддержки продвинутых технологий частиц и любой вид объёмных эффектов освещения на частицах.
	\item Туман: включает объёмный, слоистый и дальний туман для увеличения атмосферы и напряжения.
	\item Интеграция инструментальных средств: объекты и строения, которые созданы на 3ds Max или Maya, интегрированы в пределах игры и редактора.
	\item Технология PolyBump: Автономная или полностью интегрированная с другими инструментальными средствами, включая 3ds max.
	\item Скриптовая система: Базируется на популярном языке Lua. Эта удобная система позволяет установку и тонкую настройку параметров оружие/игра, проигрывание звуков и загрузку графики без использования кода C++.
	\item Модульность: Полностью написанный в модульном C++, с комментариями, документацией и разделами в множественных DLL-файлах.
	\item 	Geometry Instancing. 
\end{itemize}
