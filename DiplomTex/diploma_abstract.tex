\sectioncentered*{Реферат}
\thispagestyle{empty}
%%
%% ВНИМАНИЕ: этот реферат не соответствует СТП-01 2013
%% пример оформления реферата смотрите здесь: http://www.bsuir.by/m/12_100229_1_91132.docx 
%%

\begin{center}
Пояснительная записка \pageref*{LastPage}~с., \totfig{}~рис., \tottab{}~табл., \totref{}~источников.
\diplomabig	
\end{center}

\textbf{Ключевые слова}: построение траектории движения объектов; физические игровые движки; игровой движок Unity; моедлирование поведения игровых объектов.

\textbf{Объект проектирования:} программный модуль для вычисления и построения траектории движения игровых объектов.

\textbf{Цель проектирования}: разработка программного модуля для расчета и построения траектории движения в игровых приложениях.

Проведен анализ различных физических игровых движков. Рассмотрены математические модели построения траектории движения объектов. Рассмотрены различные физические движки и их возможности. 

Создан программный модуль с использованием среды разработки Unity и физического движка PhysX и программный модуль с использованием фреймворка libGDX и физического движка Box2D для построения траектории движения объектов в двумерном пространстве. Проведен анализ полученных модулей, описаны положительные и отрицательные стороны каждого из них. 

Создан программный модуль с использованием среды разработки Unity и физического движка PhysX для построения траектории движения объектов в трехмерном пространстве.

В разделе технико-экономического обоснования был произведён расчёт затрат на создание ПО, а также прибыли от разработки, получаемой разработчиком.

Проведенные расчеты показали экономическую целесообразность проекта.
\clearpage
